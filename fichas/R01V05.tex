% Options for packages loaded elsewhere
\PassOptionsToPackage{unicode}{hyperref}
\PassOptionsToPackage{hyphens}{url}
%
\documentclass[
]{article}
\usepackage{amsmath,amssymb}
\usepackage{iftex}
\ifPDFTeX
  \usepackage[T1]{fontenc}
  \usepackage[utf8]{inputenc}
  \usepackage{textcomp} % provide euro and other symbols
\else % if luatex or xetex
  \usepackage{unicode-math} % this also loads fontspec
  \defaultfontfeatures{Scale=MatchLowercase}
  \defaultfontfeatures[\rmfamily]{Ligatures=TeX,Scale=1}
\fi
\usepackage{lmodern}
\ifPDFTeX\else
  % xetex/luatex font selection
    \setsansfont[]{Calibri Light}
\fi

% A5 modifications begin
    \usepackage{titlesec}
    \usepackage{titling}
%    \usepackage{layouts}
    \usepackage{setspace}
    \usepackage{parskip}
    \setlength{\parskip}{0pt}

    % Ajuste del espaciado antes y después de los encabezados
    \usepackage{leading}
    
    % Configuración del interlineado para que coincida con el papel milimetrado
    \setstretch{1.25} % Ajusta este valor para que coincida con la cuadrícula del papel
    
    % Ajuste del espaciado antes y después de los encabezados
    \titlespacing*{\section}
    {0pt}{0\baselineskip}{0.5\baselineskip}
    \titlespacing*{\subsection}
    {0pt}{-0.16\baselineskip}{0\baselineskip}
    \titlespacing*{\subsubsection}
    {0pt}{0\baselineskip}{0\baselineskip}
    \titlespacing*{\paragraph}
    {0pt}{0\baselineskip}{0\baselineskip}
% A5 modificationsend


% Use upquote if available, for straight quotes in verbatim environments
\IfFileExists{upquote.sty}{\usepackage{upquote}}{}
\IfFileExists{microtype.sty}{% use microtype if available
  \usepackage[]{microtype}
  \UseMicrotypeSet[protrusion]{basicmath} % disable protrusion for tt fonts
}{}
\makeatletter
\@ifundefined{KOMAClassName}{% if non-KOMA class
  \IfFileExists{parskip.sty}{%
    \usepackage{parskip}
  }{% else
    \setlength{\parindent}{0pt}
    \setlength{\parskip}{6pt plus 2pt minus 1pt}}
}{% if KOMA class
  \KOMAoptions{parskip=half}}
\makeatother
\usepackage{xcolor}
\usepackage[a5paper, landscape, top=1.5cm, bottom=0.9cm, left=0.5cm,
right=0.5cm, headsep = 0.5cm, footskip = 12pt]{geometry}
\setlength{\emergencystretch}{3em} % prevent overfull lines
\providecommand{\tightlist}{%
  \setlength{\itemsep}{0pt}\setlength{\parskip}{0pt}}
\setcounter{secnumdepth}{-\maxdimen} % remove section numbering
\ifLuaTeX
  \usepackage{selnolig}  % disable illegal ligatures
\fi
\usepackage{bookmark}
\IfFileExists{xurl.sty}{\usepackage{xurl}}{} % add URL line breaks if available
\urlstyle{same}
\hypersetup{
  pdftitle={Funciones de medida de un vector en R},
  pdfauthor={Author Name},
  hidelinks,
  pdfcreator={LaTeX via pandoc}}

\title{Funciones de medida de un vector en R}
\author{Author Name}
\date{}

\usepackage{fancyhdr}
\pagestyle{fancy}
\fancyhead[L]{\large \textsf{Funciones de medida de un vector en R}}
\fancyhead[R]{\Large \textsc{R01V05}}
\fancyfoot[C]{\texttt{\#Estadística\slash R \#Programación\slash R
\#R\slash Vectores}
\renewcommand{\headrulewidth}{0pt}
\renewcommand*\footnoterule{}

\usepackage{background}
\usepackage{tikz}
\backgroundsetup{%
 position=current page.center,
 angle=0,
 scale=1,
 contents={%
  \begin{tikzpicture}%
    [
      normal lines/.style={blue, very thin},
      red lines/.style={red, thick},
      every node/.append style={black, align=center, opacity=1}
    ]
    \foreach \y in {0mm,3mm,9.5mm,16mm,22.5mm,29mm,35.5mm,42mm,48.5mm,55mm,61.5mm,68mm,74.5mm,81mm,87.5mm,94mm,100.5mm,107mm,113.5mm,120mm,126.5mm,148mm}
      \draw[normal lines] (0,\y) -- (8.5in,\y);
    \draw[red lines](0,133mm) -- (8.5in,133mm);
  \end{tikzpicture}%
    }}


\begin{document}
% A5 modifications begin
\let\maketitle\relax  
\leading{6.5mm} 
\setlength\baselineskip{6.5mm} 
\setlength\lineskiplimit{-\maxdimen}
% A5 modifications end

\maketitle

\hypertarget{meanx}{%
\subsubsection{\texorpdfstring{\texttt{mean(x)}}{mean(x)}}\label{meanx}}

\begin{itemize}
\tightlist
\item
  Calcula la media aritmética de las entradas del vector \(x\)
\end{itemize}

\hypertarget{diffx}{%
\subsubsection{\texorpdfstring{\texttt{diff(x)}}{diff(x)}}\label{diffx}}

\begin{itemize}
\tightlist
\item
  Calcula el vector formado por las diferencias sucesivas entre entradas
  del vector original \(x\)
\end{itemize}

\hypertarget{cumsumx}{%
\subsubsection{\texorpdfstring{\texttt{cumsum(x)}}{cumsum(x)}}\label{cumsumx}}

\begin{itemize}
\tightlist
\item
  Calcula el vector formado por las sumas acumuladas de las entradas del
  vector original \(x\).

  \begin{itemize}
  \tightlist
  \item
    Permite definir {[}{[}sucesiones{]}{]} descritas mediante
    sumatorios.
  \item
    Cada entrada de \texttt{cumsum(x)} es la suma de las entradas de
    \(x\) hasta su posición.
  \end{itemize}
\end{itemize}

\hypertarget{cumminx-cummaxx}{%
\subsubsection{\texorpdfstring{\texttt{cummin(x),\ cummax(x)}}{cummin(x), cummax(x)}}\label{cumminx-cummaxx}}

\begin{itemize}
\tightlist
\item
  Devuelven un vector con los valores mínimos o máximos encontrados en
  el vector original \(x\).
\end{itemize}

\hypertarget{cumprodx}{%
\subsubsection{\texorpdfstring{\texttt{cumprod(x)}}{cumprod(x)}}\label{cumprodx}}

\begin{itemize}
\tightlist
\item
  Calcula el vector formado por los productos acumuladas de las entradas
  del vector original \(x\).
\end{itemize}

\pagebreak

\hypertarget{referencias}{%
\section{Referencias}\label{referencias}}

\textbf{Gomilla, J. J.} (2022). \emph{Curso completo des Estadística
descriptiva - RStudio y Python. Vectores y tipos de datos en R}.
Retrieved 2024, from
\url{https://cursos.frogamesformacion.com/courses/take/estadistica-descriptiva/lessons/33618865-funciones-y-orden-de-vectores}.





\end{document}