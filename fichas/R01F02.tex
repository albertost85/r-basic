% Options for packages loaded elsewhere
\PassOptionsToPackage{unicode}{hyperref}
\PassOptionsToPackage{hyphens}{url}
%
\documentclass[
]{article}
\usepackage{amsmath,amssymb}
\usepackage{iftex}
\ifPDFTeX
  \usepackage[T1]{fontenc}
  \usepackage[utf8]{inputenc}
  \usepackage{textcomp} % provide euro and other symbols
\else % if luatex or xetex
  \usepackage{unicode-math} % this also loads fontspec
  \defaultfontfeatures{Scale=MatchLowercase}
  \defaultfontfeatures[\rmfamily]{Ligatures=TeX,Scale=1}
\fi
\usepackage{lmodern}
\ifPDFTeX\else
  % xetex/luatex font selection
    \setsansfont[]{Calibri Light}
\fi

% A5 modifications begin
    \usepackage{titlesec}
    \usepackage{titling}
%    \usepackage{layouts}
    \usepackage{setspace}
    \usepackage{parskip}
    \setlength{\parskip}{0pt}

    % Ajuste del espaciado antes y después de los encabezados
    \usepackage{leading}
    
    % Configuración del interlineado para que coincida con el papel milimetrado
    \setstretch{1.25} % Ajusta este valor para que coincida con la cuadrícula del papel
    
    % Ajuste del espaciado antes y después de los encabezados
    \titlespacing*{\section}
    {0pt}{0\baselineskip}{0.5\baselineskip}
    \titlespacing*{\subsection}
    {0pt}{-0.16\baselineskip}{0\baselineskip}
    \titlespacing*{\subsubsection}
    {0pt}{0\baselineskip}{0\baselineskip}
    \titlespacing*{\paragraph}
    {0pt}{0\baselineskip}{0\baselineskip}
    \setlength{\headheight}{18.0pt}
% A5 modificationsend


% Use upquote if available, for straight quotes in verbatim environments
\IfFileExists{upquote.sty}{\usepackage{upquote}}{}
\IfFileExists{microtype.sty}{% use microtype if available
  \usepackage[]{microtype}
  \UseMicrotypeSet[protrusion]{basicmath} % disable protrusion for tt fonts
}{}
\makeatletter
\@ifundefined{KOMAClassName}{% if non-KOMA class
  \IfFileExists{parskip.sty}{%
    \usepackage{parskip}
  }{% else
    \setlength{\parindent}{0pt}
    \setlength{\parskip}{6pt plus 2pt minus 1pt}}
}{% if KOMA class
  \KOMAoptions{parskip=half}}
\makeatother
\usepackage{xcolor}
\usepackage[a5paper, landscape, top=1.5cm, bottom=0.9cm, left=0.5cm,
right=0.5cm, headsep = 0.5cm, footskip = 12pt]{geometry}
\usepackage{color}
\usepackage{fancyvrb}
\newcommand{\VerbBar}{|}
\newcommand{\VERB}{\Verb[commandchars=\\\{\}]}
\DefineVerbatimEnvironment{Highlighting}{Verbatim}{commandchars=\\\{\}}
% Add ',fontsize=\small' for more characters per line
\usepackage{framed}
\definecolor{shadecolor}{RGB}{248,248,248}
\newenvironment{Shaded}{\begin{snugshade}}{\end{snugshade}}
\newcommand{\AlertTok}[1]{\textcolor[rgb]{0.94,0.16,0.16}{#1}}
\newcommand{\AnnotationTok}[1]{\textcolor[rgb]{0.56,0.35,0.01}{\textbf{\textit{#1}}}}
\newcommand{\AttributeTok}[1]{\textcolor[rgb]{0.13,0.29,0.53}{#1}}
\newcommand{\BaseNTok}[1]{\textcolor[rgb]{0.00,0.00,0.81}{#1}}
\newcommand{\BuiltInTok}[1]{#1}
\newcommand{\CharTok}[1]{\textcolor[rgb]{0.31,0.60,0.02}{#1}}
\newcommand{\CommentTok}[1]{\textcolor[rgb]{0.56,0.35,0.01}{\textit{#1}}}
\newcommand{\CommentVarTok}[1]{\textcolor[rgb]{0.56,0.35,0.01}{\textbf{\textit{#1}}}}
\newcommand{\ConstantTok}[1]{\textcolor[rgb]{0.56,0.35,0.01}{#1}}
\newcommand{\ControlFlowTok}[1]{\textcolor[rgb]{0.13,0.29,0.53}{\textbf{#1}}}
\newcommand{\DataTypeTok}[1]{\textcolor[rgb]{0.13,0.29,0.53}{#1}}
\newcommand{\DecValTok}[1]{\textcolor[rgb]{0.00,0.00,0.81}{#1}}
\newcommand{\DocumentationTok}[1]{\textcolor[rgb]{0.56,0.35,0.01}{\textbf{\textit{#1}}}}
\newcommand{\ErrorTok}[1]{\textcolor[rgb]{0.64,0.00,0.00}{\textbf{#1}}}
\newcommand{\ExtensionTok}[1]{#1}
\newcommand{\FloatTok}[1]{\textcolor[rgb]{0.00,0.00,0.81}{#1}}
\newcommand{\FunctionTok}[1]{\textcolor[rgb]{0.13,0.29,0.53}{\textbf{#1}}}
\newcommand{\ImportTok}[1]{#1}
\newcommand{\InformationTok}[1]{\textcolor[rgb]{0.56,0.35,0.01}{\textbf{\textit{#1}}}}
\newcommand{\KeywordTok}[1]{\textcolor[rgb]{0.13,0.29,0.53}{\textbf{#1}}}
\newcommand{\NormalTok}[1]{#1}
\newcommand{\OperatorTok}[1]{\textcolor[rgb]{0.81,0.36,0.00}{\textbf{#1}}}
\newcommand{\OtherTok}[1]{\textcolor[rgb]{0.56,0.35,0.01}{#1}}
\newcommand{\PreprocessorTok}[1]{\textcolor[rgb]{0.56,0.35,0.01}{\textit{#1}}}
\newcommand{\RegionMarkerTok}[1]{#1}
\newcommand{\SpecialCharTok}[1]{\textcolor[rgb]{0.81,0.36,0.00}{\textbf{#1}}}
\newcommand{\SpecialStringTok}[1]{\textcolor[rgb]{0.31,0.60,0.02}{#1}}
\newcommand{\StringTok}[1]{\textcolor[rgb]{0.31,0.60,0.02}{#1}}
\newcommand{\VariableTok}[1]{\textcolor[rgb]{0.00,0.00,0.00}{#1}}
\newcommand{\VerbatimStringTok}[1]{\textcolor[rgb]{0.31,0.60,0.02}{#1}}
\newcommand{\WarningTok}[1]{\textcolor[rgb]{0.56,0.35,0.01}{\textbf{\textit{#1}}}}
\setlength{\emergencystretch}{3em} % prevent overfull lines
\providecommand{\tightlist}{%
  \setlength{\itemsep}{0pt}\setlength{\parskip}{0pt}}
\setcounter{secnumdepth}{-\maxdimen} % remove section numbering
\ifLuaTeX
  \usepackage{selnolig}  % disable illegal ligatures
\fi
\usepackage{bookmark}
\IfFileExists{xurl.sty}{\usepackage{xurl}}{} % add URL line breaks if available
\urlstyle{same}
\hypersetup{
  pdftitle={Funciones para utilizar Factores en R},
  pdfauthor={Author Name},
  hidelinks,
  pdfcreator={LaTeX via pandoc}}

\title{Funciones para utilizar Factores en R}
\author{Author Name}
\date{}

\usepackage{fancyhdr}
\pagestyle{fancy}
\fancyhead[L]{\Large \textsf{Funciones para utilizar Factores en R}}
\fancyhead[R]{\Large \textsc{\jobname}}
\fancyfoot[C]{\texttt{\#Estadística\slash R \#Programación\slash R
\#R\slash factores}}
\renewcommand{\headrulewidth}{0pt}
\renewcommand*\footnoterule{}

\usepackage{background}
\usepackage{tikz}
\backgroundsetup{%
 position=current page.center,
 angle=0,
 scale=1,
 contents={%
  \begin{tikzpicture}%
    [
      normal lines/.style={blue, very thin},
      red lines/.style={red, thick},
      every node/.append style={black, align=center, opacity=1}
    ]
    \foreach \y in {0mm,3mm,9.5mm,16mm,22.5mm,29mm,35.5mm,42mm,48.5mm,55mm,61.5mm,68mm,74.5mm,81mm,87.5mm,94mm,100.5mm,107mm,113.5mm,120mm,126.5mm,148mm}
      \draw[normal lines] (0,\y) -- (8.5in,\y);
    \draw[red lines](0,133mm) -- (8.5in,133mm);
  \end{tikzpicture}%
    }}

\DefineVerbatimEnvironment{verbatim}{Verbatim}{%
  formatcom={\baselineskip=-\maxdimen\lineskip=0pt}
}

\begin{document}
% A5 modifications begin
\let\maketitle\relax  
\leading{6.5mm} 
\setlength\baselineskip{6.5mm} 
\setlength\lineskiplimit{-\maxdimen}
% A5 modifications end

\maketitle

Para definir un factor, primero hemos de definir un vector y
transformarlo por medio de una de las funciones \texttt{factor()} o
\texttt{as.factor()}.

\hypertarget{factorvectorlevels...}{%
\subsubsection{\texorpdfstring{\texttt{factor(vector,levels=...)}}{factor(vector,levels=...)}}\label{factorvectorlevels...}}

\begin{itemize}
\tightlist
\item
  Define un factor a partir del vector y dispone de algunos parámetros
  que permiten modificar el factor que se crea:

  \begin{itemize}
  \tightlist
  \item
    \texttt{levels}: permite especificar los niveles e incluso añadir
    niveles que no aparecen en el vector
  \item
    \texttt{labels}: permite cambiar los nombres de los niveles
  \end{itemize}
\end{itemize}

\hypertarget{as.factorvector}{%
\subsubsection{\texorpdfstring{\texttt{as.factor(vector)}}{as.factor(vector)}}\label{as.factorvector}}

\begin{itemize}
\tightlist
\item
  Define un factor a partir del vector tomando sus valores como niveles.
\end{itemize}

\hypertarget{levelsfactor}{%
\subsubsection{\texorpdfstring{\texttt{levels(factor)}}{levels(factor)}}\label{levelsfactor}}

\begin{itemize}
\tightlist
\item
  Devuelve los niveles del factor, y permite cambiarlos.
\end{itemize}

\hypertarget{ejemplo}{%
\subsection{Ejemplo}\label{ejemplo}}

\begin{Shaded}
\begin{Highlighting}[]
\NormalTok{fac }\OtherTok{=} \FunctionTok{factor}\NormalTok{(}\FunctionTok{c}\NormalTok{(}\DecValTok{1}\NormalTok{,}\DecValTok{1}\NormalTok{,}\DecValTok{1}\NormalTok{,}\DecValTok{2}\NormalTok{,}\DecValTok{2}\NormalTok{,}\DecValTok{3}\NormalTok{,}\DecValTok{2}\NormalTok{,}\DecValTok{4}\NormalTok{,}\DecValTok{1}\NormalTok{,}\DecValTok{3}\NormalTok{,}\DecValTok{3}\NormalTok{,}\DecValTok{4}\NormalTok{,}\DecValTok{2}\NormalTok{,}\DecValTok{3}\NormalTok{,}\DecValTok{4}\NormalTok{,}\DecValTok{4}\NormalTok{), }
       \AttributeTok{levels =} \FunctionTok{c}\NormalTok{(}\DecValTok{1}\NormalTok{,}\DecValTok{2}\NormalTok{,}\DecValTok{3}\NormalTok{,}\DecValTok{4}\NormalTok{,}\DecValTok{5}\NormalTok{), }\AttributeTok{labels =} \FunctionTok{c}\NormalTok{(}\StringTok{"Sus"}\NormalTok{,}\StringTok{"Apr"}\NormalTok{,}\StringTok{"Not"}\NormalTok{,}\StringTok{"Exc"}\NormalTok{,}\StringTok{"NOSE"}\NormalTok{))}
\NormalTok{fac}
\end{Highlighting}
\end{Shaded}

\begin{verbatim}
##  [1] Sus Sus Sus Apr Apr Not Apr Exc Sus Not Not Exc Apr Not Exc Exc
## Levels: Sus Apr Not Exc NOSE
\end{verbatim}

\begin{Shaded}
\begin{Highlighting}[]
\FunctionTok{levels}\NormalTok{(fac) }\OtherTok{=} \FunctionTok{c}\NormalTok{(}\StringTok{"Sus"}\NormalTok{,}\StringTok{"Apr"}\NormalTok{,}\StringTok{"Sob"}\NormalTok{,}\StringTok{"Sob"}\NormalTok{,}\StringTok{"NOSE"}\NormalTok{)}
\NormalTok{fac}
\end{Highlighting}
\end{Shaded}

\begin{verbatim}
##  [1] Sus Sus Sus Apr Apr Sob Apr Sob Sus Sob Sob Sob Apr Sob Sob Sob
## Levels: Sus Apr Sob NOSE
\end{verbatim}

\pagebreak

\hypertarget{referencias}{%
\section{Referencias}\label{referencias}}

\textbf{Gomilla, J. J.} (2022). \emph{Curso completo des Estadística
descriptiva - RStudio y Python. Vectores y tipos de datos en R}.
Retrieved 2024, from
\url{https://cursos.frogamesformacion.com/courses/take/estadistica-descriptiva/lessons/33618868-factores}.





\end{document}