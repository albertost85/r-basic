% Options for packages loaded elsewhere
\PassOptionsToPackage{unicode}{hyperref}
\PassOptionsToPackage{hyphens}{url}
%
\documentclass[
]{article}
\usepackage{amsmath,amssymb}
\usepackage{iftex}
\ifPDFTeX
  \usepackage[T1]{fontenc}
  \usepackage[utf8]{inputenc}
  \usepackage{textcomp} % provide euro and other symbols
\else % if luatex or xetex
  \usepackage{unicode-math} % this also loads fontspec
  \defaultfontfeatures{Scale=MatchLowercase}
  \defaultfontfeatures[\rmfamily]{Ligatures=TeX,Scale=1}
\fi
\usepackage{lmodern}
\ifPDFTeX\else
  % xetex/luatex font selection
    \setsansfont[]{Calibri Light}
\fi

% A5 modifications begin
    \usepackage{titlesec}
    \usepackage{titling}
%    \usepackage{layouts}
    \usepackage{setspace}
    \usepackage{parskip}
    \setlength{\parskip}{0pt}

    % Ajuste del espaciado antes y después de los encabezados
    \usepackage{leading}
    
    % Configuración del interlineado para que coincida con el papel milimetrado
    \setstretch{1.25} % Ajusta este valor para que coincida con la cuadrícula del papel
    
    % Ajuste del espaciado antes y después de los encabezados
    \titlespacing*{\section}
    {0pt}{0\baselineskip}{0.5\baselineskip}
    \titlespacing*{\subsection}
    {0pt}{-0.16\baselineskip}{0\baselineskip}
    \titlespacing*{\subsubsection}
    {0pt}{0\baselineskip}{0\baselineskip}
    \titlespacing*{\paragraph}
    {0pt}{0\baselineskip}{0\baselineskip}
    \setlength{\headheight}{18.0pt}
% A5 modificationsend


% Use upquote if available, for straight quotes in verbatim environments
\IfFileExists{upquote.sty}{\usepackage{upquote}}{}
\IfFileExists{microtype.sty}{% use microtype if available
  \usepackage[]{microtype}
  \UseMicrotypeSet[protrusion]{basicmath} % disable protrusion for tt fonts
}{}
\makeatletter
\@ifundefined{KOMAClassName}{% if non-KOMA class
  \IfFileExists{parskip.sty}{%
    \usepackage{parskip}
  }{% else
    \setlength{\parindent}{0pt}
    \setlength{\parskip}{6pt plus 2pt minus 1pt}}
}{% if KOMA class
  \KOMAoptions{parskip=half}}
\makeatother
\usepackage{fancyvrb}
\usepackage{xcolor}
\usepackage[a5paper, landscape, top=1.5cm, bottom=0.9cm, left=0.5cm,
right=0.5cm, headsep = 0.5cm, footskip = 12pt]{geometry}
\usepackage{color}
\usepackage{fancyvrb}
\newcommand{\VerbBar}{|}
\newcommand{\VERB}{\Verb[commandchars=\\\{\}]}
\DefineVerbatimEnvironment{Highlighting}{Verbatim}{commandchars=\\\{\}}
% Add ',fontsize=\small' for more characters per line
\usepackage{framed}
\definecolor{shadecolor}{RGB}{248,248,248}
\newenvironment{Shaded}{\begin{snugshade}}{\end{snugshade}}
\newcommand{\AlertTok}[1]{\textcolor[rgb]{0.94,0.16,0.16}{#1}}
\newcommand{\AnnotationTok}[1]{\textcolor[rgb]{0.56,0.35,0.01}{\textbf{\textit{#1}}}}
\newcommand{\AttributeTok}[1]{\textcolor[rgb]{0.13,0.29,0.53}{#1}}
\newcommand{\BaseNTok}[1]{\textcolor[rgb]{0.00,0.00,0.81}{#1}}
\newcommand{\BuiltInTok}[1]{#1}
\newcommand{\CharTok}[1]{\textcolor[rgb]{0.31,0.60,0.02}{#1}}
\newcommand{\CommentTok}[1]{\textcolor[rgb]{0.56,0.35,0.01}{\textit{#1}}}
\newcommand{\CommentVarTok}[1]{\textcolor[rgb]{0.56,0.35,0.01}{\textbf{\textit{#1}}}}
\newcommand{\ConstantTok}[1]{\textcolor[rgb]{0.56,0.35,0.01}{#1}}
\newcommand{\ControlFlowTok}[1]{\textcolor[rgb]{0.13,0.29,0.53}{\textbf{#1}}}
\newcommand{\DataTypeTok}[1]{\textcolor[rgb]{0.13,0.29,0.53}{#1}}
\newcommand{\DecValTok}[1]{\textcolor[rgb]{0.00,0.00,0.81}{#1}}
\newcommand{\DocumentationTok}[1]{\textcolor[rgb]{0.56,0.35,0.01}{\textbf{\textit{#1}}}}
\newcommand{\ErrorTok}[1]{\textcolor[rgb]{0.64,0.00,0.00}{\textbf{#1}}}
\newcommand{\ExtensionTok}[1]{#1}
\newcommand{\FloatTok}[1]{\textcolor[rgb]{0.00,0.00,0.81}{#1}}
\newcommand{\FunctionTok}[1]{\textcolor[rgb]{0.13,0.29,0.53}{\textbf{#1}}}
\newcommand{\ImportTok}[1]{#1}
\newcommand{\InformationTok}[1]{\textcolor[rgb]{0.56,0.35,0.01}{\textbf{\textit{#1}}}}
\newcommand{\KeywordTok}[1]{\textcolor[rgb]{0.13,0.29,0.53}{\textbf{#1}}}
\newcommand{\NormalTok}[1]{#1}
\newcommand{\OperatorTok}[1]{\textcolor[rgb]{0.81,0.36,0.00}{\textbf{#1}}}
\newcommand{\OtherTok}[1]{\textcolor[rgb]{0.56,0.35,0.01}{#1}}
\newcommand{\PreprocessorTok}[1]{\textcolor[rgb]{0.56,0.35,0.01}{\textit{#1}}}
\newcommand{\RegionMarkerTok}[1]{#1}
\newcommand{\SpecialCharTok}[1]{\textcolor[rgb]{0.81,0.36,0.00}{\textbf{#1}}}
\newcommand{\SpecialStringTok}[1]{\textcolor[rgb]{0.31,0.60,0.02}{#1}}
\newcommand{\StringTok}[1]{\textcolor[rgb]{0.31,0.60,0.02}{#1}}
\newcommand{\VariableTok}[1]{\textcolor[rgb]{0.00,0.00,0.00}{#1}}
\newcommand{\VerbatimStringTok}[1]{\textcolor[rgb]{0.31,0.60,0.02}{#1}}
\newcommand{\WarningTok}[1]{\textcolor[rgb]{0.56,0.35,0.01}{\textbf{\textit{#1}}}}
\setlength{\emergencystretch}{3em} % prevent overfull lines
\providecommand{\tightlist}{%
  \setlength{\itemsep}{0pt}\setlength{\parskip}{0pt}}
\setcounter{secnumdepth}{-\maxdimen} % remove section numbering
\ifLuaTeX
  \usepackage{selnolig}  % disable illegal ligatures
\fi
\usepackage{bookmark}
\IfFileExists{xurl.sty}{\usepackage{xurl}}{} % add URL line breaks if available
\urlstyle{same}
\VerbatimFootnotes % allow verbatim text in footnotes
\hypersetup{
  pdftitle={Función para obtener valores propios de una matriz en R},
  pdfauthor={Author Name},
  hidelinks,
  pdfcreator={LaTeX via pandoc}}

\title{Función para obtener valores propios de una matriz en R}
\author{Author Name}
\date{}

\usepackage{fancyhdr}
\pagestyle{fancy}
\fancyhead[L]{\Large \textsf{Función para obtener valores propios de una
matriz en R}}
\fancyhead[R]{\Large \textsc{\jobname}}
\fancyfoot[C]{\texttt{\#Estadística\slash R \#Programación\slash R
\#R\slash Matrices}}
\renewcommand{\headrulewidth}{0pt}
\renewcommand*\footnoterule{}

\DefineVerbatimEnvironment{verbatim}{Verbatim}{%
  formatcom={\baselineskip=-\maxdimen\lineskip=0pt}
}

\begin{document}
% A5 modifications begin
\let\maketitle\relax  
\leading{6.5mm} 
\setlength\baselineskip{6.5mm} 
\setlength\lineskiplimit{-\maxdimen}
% A5 modifications end

\maketitle

\hypertarget{eigenmatriz}{%
\subsubsection{\texorpdfstring{\texttt{eigen(matriz)}}{eigen(matriz)}}\label{eigenmatriz}}

\begin{itemize}
\tightlist
\item
  Calcula los valores \textbf{vaps} y vectores propios \textbf{veps}
\end{itemize}

Si hay algún vap con multiplicidad algebraica mayor que 1 (es decir, que
aparece más de una vez), la función \texttt{eigen()} da tantos valores
de este vap como su multiplicidad algebraica indica. Además, en este
caso, R intenta que los veps asociados a cada uno de estos vaps sean
\textbf{linealmente independientes} {[}\textbf{R01A02}{]}. Por tanto,
cuando como resultado obtenemos veps repetidos asociados a un vap de
multiplicidad algebraica mayor que 1, es porque para este vap no existen
tantos veps linealmente independientes como su multiplicidad algebraica
y, por consiguiente, la matriz no es \textbf{diagonalizable}
\textbf{R01A03}.

\hypertarget{eigenmatrizvalues}{%
\subsubsection{\texorpdfstring{\texttt{eigen(matriz)\$values}}{eigen(matriz)\$values}}\label{eigenmatrizvalues}}

\begin{itemize}
\tightlist
\item
  Nos da el vector con los vaps de la matriz en orden decreciente de su
  valor absoluto y repetidos tantas veces como su multiplicidad
  algebraica.
\end{itemize}

\hypertarget{eigenmatrizvectors}{%
\subsubsection{\texorpdfstring{\texttt{eigen(matriz)\$vectors}}{eigen(matriz)\$vectors}}\label{eigenmatrizvectors}}

\begin{itemize}
\tightlist
\item
  Nos da una matriz cuyas columnas son los veps de la matriz.
\end{itemize}

\hypertarget{ejemplo-r01a01-descomposiciuxf3n-canuxf3nica-de-matrices-en-r}{%
\subsection{\texorpdfstring{Ejemplo: {[}{[}\textbf{R01A01}
\textbar Descomposición canónica de matrices en
R{]}{]}}{Ejemplo: {[}{[}R01A01 \textbar Descomposición canónica de matrices en R{]}{]}}}\label{ejemplo-r01a01-descomposiciuxf3n-canuxf3nica-de-matrices-en-r}}

\hypertarget{ejemplo}{%
\subsection{Ejemplo:}\label{ejemplo}}

\begin{Shaded}
\begin{Highlighting}[]
\NormalTok{M }\OtherTok{=} \FunctionTok{matrix}\NormalTok{(}\FunctionTok{c}\NormalTok{(}\DecValTok{0}\NormalTok{,}\DecValTok{1}\NormalTok{,}\DecValTok{0}\NormalTok{,}\SpecialCharTok{{-}}\DecValTok{7}\NormalTok{,}\DecValTok{3}\NormalTok{,}\SpecialCharTok{{-}}\DecValTok{1}\NormalTok{,}\DecValTok{16}\NormalTok{,}\SpecialCharTok{{-}}\DecValTok{3}\NormalTok{,}\DecValTok{4}\NormalTok{), }\AttributeTok{nrow =} \DecValTok{3}\NormalTok{, }\AttributeTok{byrow =} \ConstantTok{TRUE}\NormalTok{)}
\FunctionTok{eigen}\NormalTok{(M)}
\end{Highlighting}
\end{Shaded}

\begin{verbatim}
## eigen() decomposition
## $values
## [1] 3 2 2
## 
## $vectors
##            [,1]       [,2]       [,3]
## [1,] -0.1301889 -0.1825742 -0.1825742
## [2,] -0.3905667 -0.3651484 -0.3651484
## [3,]  0.9113224  0.9128709  0.9128709
\end{verbatim}

\pagebreak

\hypertarget{referencias}{%
\section{Referencias}\label{referencias}}

\textbf{Gomilla, J. J.} (2022). \emph{Curso completo des Estadística
descriptiva - RStudio y Python. Vectores y tipos de datos en R}.
Retrieved 2024, from
\url{https://cursos.frogamesformacion.com/courses/take/estadistica-descriptiva/lessons/33618869-un-repaso-de-algebra-lineal}.




\end{document}